% !Mode:: "TeX:DE:UTF-8:Main"

\documentclass[parskip=half-]{scrartcl}
\usepackage{combofont}
\usepackage{ydoc-code,ydoc-desc}
\usepackage{fontspec}

\usepackage{xcolor}


\setupcombofont{combotest-regular}
 {
  {file:lmroman10-regular.otf:\combodefaultfeat} at #1pt,
  {file:lmsans10-regular.otf} at \fpeval{#1/10*15}pt,
  {file:cmunrm.otf}           at #1pt
 }
 {
   {} ,
   0x41-0x5A*0x21*0x3F,
   fallback
 }

\setupcombofont{combotest-bold}
 {
  {file:lmroman10-bold.otf:\combodefaultfeat} at #1pt,
  {file:lmsans10-bold.otf} at \fpeval{#1/10*15}pt,
  {file:cmunbbx.otf}       at #1pt
 }
 {
   {} , 
   0x41-0x5A*0x21*0x3F, 
   fallback
 }



\DeclareFontFamily{TU}{combotest}{}
\DeclareFontShape{TU} {combotest}{m}{n}{<->combo*combotest-regular}{}
\DeclareFontShape{TU} {combotest}{bx}{n}{<->combo*combotest-bold}{}

\title{The \texttt{combofont} package}
\author{Ulrike Fischer\thanks{fischer@troubleshooting-tex.de}}
\newcommand\package[1]{\texttt{#1}}
\begin{document}
\maketitle

\section{Status: EXPERIMENTAL}

This is a EXPERIMENTAL package. 

It can disappear without notice e.\,g. if the \package{luaotfload} changes so that it no longer work, or if luatex changes, or if \package{fontspec} includes the code.

It is also possible that syntax and commands change in a incompatible way. So if you use it in a production environment: \textbf{You have been warned}

\section{Introduction}
In version 2.7. \package{luaotfload} supports combining characters from multiple fonts into a single virtualized one.

That means that one can build a font that takes e.g. the capital letters from a sans serif font and the lowercase letters from a serif font. Or a font that pulls in missing greek or cyrillic glyphs from another font.

The methods pulls in \emph{only} glyphs. It is not suitable for every imaginable font combination -- some drawbacks are described below -- and one should use it with care. Nevertheless it is a quite neat extension of the tools to manipulate fonts.

The main problem with the examples in the \package{luaotfload} manual is that it creates fonts of a fix size. This means that they don't respond to command like \verb+\large+ or \verb+\footnotesize+. 

After trying around a bit and then asking a question (https://tex.stackexchange.com/questions/371647/call-a-luatex-combo-font-through-nfss) I got from David Carlisle the idea to use a \texttt{size}-Funktion which one define with \verb+\DeclareSizeFunction+ to inject the needed code to size the combo-font in a nfss-\verb+\DeclareFontShape+-command.

\package{combofont.sty} is the result. 

\section{Using combo fonts}

To be able to use a combo font with standard \LaTeX\ font commands you have to do two things:

\begin{enumerate}
\item Setup and describe the building of the combo font with \verb+\setupcombofont+

\item Write \texttt{nfss}-declarations 
\end{enumerate}

\subsection{Setup the combo font}

\DescribeMacro\setupcombofont{<name>}{<comma list of basefonts>}{<comma list of ranges and code-points>}

\begin{description}
  \item[\marg{name}] is the name of the font. It should be some unique ascii-string without spaces. If you intent to define lots of fonts it would be a good idea to think about a sensible naming sheme. In the example here I simply used \texttt{combotest-regular} and \texttt{combotest-bold}.
      
  \item[\marg{comma list of basefonts}] This should be a list of font declarations you want to use to build your combo font. The syntax used is described in the \package{luaotfload} manual. Example:
      
\begin{verbatim}
{
 {file:lmroman10-regular.otf:\combodefaultfeat} at #1pt,
 {file:lmsans10-regular.otf} at \fpeval{#1/10*15}pt,
 {file:cmunrm.otf}           at #1pt
}
\end{verbatim}

Important points are:
\begin{description}
\item[Order of the fonts] The first font is the main font which will receive the glyphs. So think carefully which font is should be and setup its font features correctly. \package{combofont} defines as a helper command \verb+\combodefaultfeat+ which sets \texttt{mode=node;script=latn;language=DFLT;+tlig;}. 
    
\item[Size declaration] The font description should end with a size declaration line \verb+at #1pt+. When processing the font \verb+#1+ will be replaced by the current font size. As you can see in the second font you can do calculations.
\end{description}

\item[\marg{comma list of ranges and code-points}] This is a comma list of settings which describe which glyphs are taken from the respective font. Example:

\begin{verbatim}
{
   {} ,
   0x41-0x5A*0x21*0x3F,
   fallback
}
\end{verbatim}

Important points
\begin{enumerate}
\item there should be as many settings as there are fonts. 
\item empty entries should be marked with a pair ob braces (normally this should done for the first font).
\item You can add ranges of code points and single code points. Blocks are separated by an asterix \verb+*+. 
\item The keyword \texttt{fallback} means that this font is used for „missing glyphs“  
\end{enumerate}  
\end{description}


\section{Call through nfss?}

%\font\test={file:lmroman10-regular.otf:+smcp}\test abc

\fontfamily{combotest}\selectfont --
Some Text with Capital Words!
Eh bien, mon prince. Gênes et Lueques ne sont plus que des
apanages, des поместья, de la famille Buonaparte?
%
\large
Some Text with Capital Words!
Eh bien, mon prince. Gênes et Lueques ne sont plus que des
apanages, des поместья, de la famille Buonaparte?

\tiny

Some Text with Capital Words!
Eh bien, mon prince. Gênes et Lueques ne sont plus que des
apanages, des поместья, de la famille Buonaparte?

\bfseries
\normalsize

Some Text with Capital Words!
Eh bien, mon prince. Gênes et Lueques ne sont plus que des
apanages, des поместья, de la famille Buonaparte?
%
\large
Some Text with Capital Words!
Eh bien, mon prince. Gênes et Lueques ne sont plus que des
apanages, des поместья, de la famille Buonaparte?

\tiny

Some Text with Capital Words fi
Eh bien, mon prince. Gênes et Lueques ne sont plus que des
apanages, des поместья, de la famille Buonaparte êßä
%


\end{document}



\ExplSyntaxOn
\__combo_call_basefonts:n {mycombofont}

\l_combo_tmpfont_i_tl abc

\l_combo_tmpfont_ii_tl abc

%\tl_show:N \l__combo_mycombofont_combodesc_tl
 \font\testfont =\l__combo_mycombofont_combodesc_tl


 \testfont Abc

 %\edef\test{\noexpand\font\noexpand\mytestfont = \unexpanded\expandafter{\l__combo_mycombofont_combodesc_tl}}
% \test abc

\ExplSyntaxOff


> external@font=macro:
->"combo: 1 -> fontid one ; 2 -> fontid two , 0x41-0x5A;".

> external@font=macro:
->"combo:1->fontid l_combo_tmpfont_i_tl ;2->fontid l_combo_tmpfont_ii_tl ;".

\documentclass[parskip=half-]{scrartcl}
\usepackage{fontspec,xfp}

\newcommand\requestmycombofont[1]{%
\font \one = {file:lmmono10-regular.otf} at \fpeval{#1*10}pt
\font \two = {file:lmsans10-regular.otf} at \fpeval{#1*15}pt
\def\mycombofont{"combo: 1 -> \fontid \one ;
2 -> \fontid \two , 0x41-0x5A;"}
}

\makeatletter
\DeclareSizeFunction{ulrike}{\ulrike@sfcnt\@font@warning}
\def\ulrike@sfcnt#1{%
\requestmycombofont{\fpeval{\f@size/10}}%
\expandafter\def\expandafter\external@font\expandafter{\mycombofont}
}%

\DeclareFontFamily{TU}{xxx}{}

\DeclareFontShape{TU}{xxx}{m}{n}{<->ulrike*zzz}{}



\begin{document}

\section{Call through nfss?}
\normalsize
\fontfamily{xxx}\selectfont
Some Text with Capital Words

\large
Some Text with Capital Words
\end{document}

\documentclass[parskip=half-]{scrartcl}
\usepackage{fontspec,xfp}

\newcommand\requestmycombofont[1]{%
\font \one = {file:lmmono10-regular.otf} at \fpeval{#1*10}pt
\font \two = {file:lmsans10-regular.otf} at \fpeval{#1*15}pt
\def\mycombofont{"combo: 1 -> \fontid \one ;
2 -> \fontid \two , 0x41-0x5A;"}
}

\makeatletter
\DeclareSizeFunction{ulrike}{\ulrike@sfcnt\@font@warning}
\def\ulrike@sfcnt#1{%
\requestmycombofont{\fpeval{\f@size/10}}%
\expandafter\def\expandafter\external@font\expandafter{\mycombofont}
\show\external@font
}%

\DeclareFontFamily{TU}{xxx}{}

\DeclareFontShape{TU}{xxx}{m}{n}{<->ulrike*zzz}{}



\begin{document}

\section{Call through nfss?}
\normalsize
\fontfamily{xxx}\selectfont
Some Text with Capital Words

\large
Some Text with Capital Words
\end{document} 